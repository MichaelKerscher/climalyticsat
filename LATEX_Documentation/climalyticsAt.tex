\documentclass[sigconf]{acmart}

\title{Climalytics AT}

\author{David Kalteis}
\email{s2410455001@fhooe.at}
\affiliation{\institution{FH Hagenberg}\department{Mobile Computing}\country{Austria}}

\author{Dominik Forsthuber}
\email{s2410455011@fhooe.at}
\affiliation{\institution{FH Hagenberg}\department{Mobile Computing}\country{Austria}}

\author{Michael Kerscher}
\email{s2410455014@fhooe.at}
\affiliation{\institution{FH Hagenberg}\department{Mobile Computing}\country{Austria}}

\begin{document}

\begin{abstract}
Your abstract here.
\end{abstract}

\maketitle

\section{Introduction}
Our project focuses on analyzing long-term weather trends and extreme climate patterns in Austria using scalable Big Data technologies. We aim to uncover patterns in temperature, precipitation, and extreme events across regions and over time.

\section{Dataset}
This dataset contains monthly aggregated climate data from various weather stations across Austria. Each entry includes numerous meteorological measurements (e.g., temperature extremes, precipitation, humidity, sunshine duration, frost days, wind data) over many years, making it suitable for large-scale time series and spatial weather analysis.
Time span: January 1970 – April 2025 (monthly resolution) \linebreak
The dataset used in this study was downloaded from GeoSphere Austria's climate data portal~\cite{geosphere_klima_v2_1m}.\\

\textbf{Downloaded datafiles:} 
\begin{itemize}
    \item \verb|climate_all_stations.csv    676 MB |
    \item \verb|parameter_metadata.csv       58 KB |
    \item \verb|stations_metadata.csv       168 KB |
\end{itemize}

\section{Research questions}
\subsection{How does the long-term trend in mean monthly temperature vary with elevation?}

\begin{figure}[H]
  \centering
  \includegraphics[width=0.4\linewidth]{img/placeholder.png}
  \caption{Placeholder figure.}
  \label{fig:placeholder}
\end{figure}


\subsection{Which geographic zones (valleys, plateaus, alpine corridors) show the largest shifts in “hot days” (<= 30 C) and “frost days” (>= 0 C) since 1970?}
\subsection{How do station installation dates and validity periods create spatio-temporal gaps, and where are the largest “data deserts”?}
\subsection{Which seasonal windows and locations optimize safety—combining sunshine hours, wind-gust flags, and frost/heat indicators?}

\subsubsection{Calculation Method}

The safety score is a composite metric designed to evaluate the environmental safety conditions for each weather station across Austria. It combines the following variables:

\begin{itemize}
    \item \textbf{Average Sunshine Hours}: Higher values are considered favorable.
    \item \textbf{Frequency of Wind Gusts}: Higher values are penalized due to increased hazard potential.
    \item \textbf{Frequency of Frost Days}: Higher values indicate harsher conditions and are penalized.
    \item \textbf{Frequency of Heat Days}: Moderately penalized to reflect discomfort and potential risk.
\end{itemize}

The score is calculated as a weighted sum:

\begin{equation}
\text{Score} = \text{Sunshine} - 2 \cdot \text{Wind Gust Freq} - 2 \cdot \text{Frost Freq} - 1.5 \cdot \text{Heat Freq}
\end{equation}

A higher safety score indicates more favorable and stable environmental conditions.

\subsubsection{Top Station Scores}

Table~\ref{tab:top_scores} shows the top 10 station-season combinations with the highest safety scores. These stations are located exclusively in high alpine zones (\textgreater 2000\,m) within Salzburg and exhibit high sunshine hours, but still significant wind and frost frequencies.

\begin{table}[H]
\centering
\begin{tabular}{|c|c|c|c|c|}
\hline
\textbf{Season} & \textbf{Station} & \textbf{Region} & \textbf{Sunshine [h]} & \textbf{Score} \\
\hline
Spring & 15410 & Salzburg & 461.3 & 398.8 \\
Spring & 213   & Salzburg & 440.8 & 378.2 \\
Spring & 15411 & Salzburg & 394.1 & 331.2 \\
Summer & 15410 & Salzburg & 336.1 & 291.5 \\
Spring & 15322 & Salzburg & 340.9 & 283.9 \\
Summer & 213   & Salzburg & 319.8 & 278.3 \\
Summer & 15411 & Salzburg & 279.7 & 243.9 \\
Winter & 15410 & Salzburg & 303.8 & 239.3 \\
Winter & 213   & Salzburg & 296.1 & 232.3 \\
Winter & 15411 & Salzburg & 294.0 & 231.4 \\
\hline
\end{tabular}
\caption{Top 10 Safety Scores by Station and Season}
\label{tab:top_scores}
\end{table}

\subsubsection{Visual Interpretation}

\begin{figure}[H]
    \centering
    \includegraphics[width=0.45\textwidth]{img/sunshine_by_season_zone.png}
    \caption{Average Sunshine Hours by Season and Elevation Zone}
    \label{fig:sunshine}
\end{figure}

Figure~\ref{fig:sunshine} shows that the high alpine zone consistently records higher average sunshine hours across all seasons, especially in spring. This contributes positively to their safety scores.

\begin{figure}[H]
    \centering
    \includegraphics[width=0.45\textwidth]{img/safety_score_boxplot.png}
    \caption{Safety Score Distribution by Season and Bundesland}
    \label{fig:safety_boxplot}
\end{figure}

Figure~\ref{fig:safety_boxplot} presents the distribution of safety scores. Salzburg, Tirol, and Vorarlberg show the highest spread and highest values in winter and spring, corresponding to high-elevation stations.

\begin{figure}[H]
    \centering
    \includegraphics[width=0.45\textwidth]{img/station_map.png}
    \caption{Map of Stations by Safety Score (Green: High, Red: Low)}
    \label{fig:station_map}
\end{figure}

Figure~\ref{fig:station_map} visualizes the spatial distribution of safety scores. Green points indicate stations with relatively high safety scores, found predominantly in the Alpine regions. Red points represent stations with lower scores, more frequent in lower elevation areas.

\section{Conclusion}

\bibliographystyle{ACM-Reference-Format}
\bibliography{software}

\end{document}